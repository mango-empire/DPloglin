\documentclass{article}\usepackage[]{graphicx}\usepackage[]{xcolor}
% maxwidth is the original width if it is less than linewidth
% otherwise use linewidth (to make sure the graphics do not exceed the margin)
\makeatletter
\def\maxwidth{ %
  \ifdim\Gin@nat@width>\linewidth
    \linewidth
  \else
    \Gin@nat@width
  \fi
}
\makeatother

\definecolor{fgcolor}{rgb}{0.345, 0.345, 0.345}
\newcommand{\hlnum}[1]{\textcolor[rgb]{0.686,0.059,0.569}{#1}}%
\newcommand{\hlstr}[1]{\textcolor[rgb]{0.192,0.494,0.8}{#1}}%
\newcommand{\hlcom}[1]{\textcolor[rgb]{0.678,0.584,0.686}{\textit{#1}}}%
\newcommand{\hlopt}[1]{\textcolor[rgb]{0,0,0}{#1}}%
\newcommand{\hlstd}[1]{\textcolor[rgb]{0.345,0.345,0.345}{#1}}%
\newcommand{\hlkwa}[1]{\textcolor[rgb]{0.161,0.373,0.58}{\textbf{#1}}}%
\newcommand{\hlkwb}[1]{\textcolor[rgb]{0.69,0.353,0.396}{#1}}%
\newcommand{\hlkwc}[1]{\textcolor[rgb]{0.333,0.667,0.333}{#1}}%
\newcommand{\hlkwd}[1]{\textcolor[rgb]{0.737,0.353,0.396}{\textbf{#1}}}%
\let\hlipl\hlkwb

\usepackage{framed}
\makeatletter
\newenvironment{kframe}{%
 \def\at@end@of@kframe{}%
 \ifinner\ifhmode%
  \def\at@end@of@kframe{\end{minipage}}%
  \begin{minipage}{\columnwidth}%
 \fi\fi%
 \def\FrameCommand##1{\hskip\@totalleftmargin \hskip-\fboxsep
 \colorbox{shadecolor}{##1}\hskip-\fboxsep
     % There is no \\@totalrightmargin, so:
     \hskip-\linewidth \hskip-\@totalleftmargin \hskip\columnwidth}%
 \MakeFramed {\advance\hsize-\width
   \@totalleftmargin\z@ \linewidth\hsize
   \@setminipage}}%
 {\par\unskip\endMakeFramed%
 \at@end@of@kframe}
\makeatother

\definecolor{shadecolor}{rgb}{.97, .97, .97}
\definecolor{messagecolor}{rgb}{0, 0, 0}
\definecolor{warningcolor}{rgb}{1, 0, 1}
\definecolor{errorcolor}{rgb}{1, 0, 0}
\newenvironment{knitrout}{}{} % an empty environment to be redefined in TeX

\usepackage{alltt}
\usepackage{graphicx} % Required for inserting images
\usepackage{amsmath}
\usepackage{amsfonts}
\usepackage{statmath}
\usepackage{biblatex}
\usepackage{amsthm}
\theoremstyle{definition}
\newtheorem{definition}{Definition}[section]

\newcommand{\dpn}{\textbf{mcDP}}

\title{Data Augmentation for Privacy Aware Analysis}
\author{DP Group}
\date{September 2023}
\IfFileExists{upquote.sty}{\usepackage{upquote}}{}
\begin{document}

\maketitle

\begin{abstract}
    This paper serves as a reference and introduction on using the $\dpn$ R
    package. This package provides a frame work for adapting an existing
    Bayesian analysis to account for privatized data. A key strength of
    this framework is the ability to target the exact posterior in settings
    where the likelihood is too complex to analytically express. The main
    purpose of the package is to ingest three model components and return
    a Markov chain targeting the posterior given the privatized data. These
    model components are the (1) naive posterior sampler (2) likelihood sampler
    and (3) privacy mechanism. Other optional components can be specified to
    speed up calculations. Several working examples are given.
\end{abstract}

\section*{Introduction}

Blah blah introduce differential privacy.

\section*{Simple Binomial Proportion}
Reference the transportation example. Discrete Gaussian distribution

\begin{definition}
Let $\mu, \sigma \in \mathbb{R}$  and $\sigma > 0$. The discrete Gaussian
distribution has probability distribution
\begin{align*}
P(X = x) &= \dfrac{\exp(-(x-\mu)^2/2\sigma^2)}{\sum_{y \in \mathbb{Z}} \exp(-(y-\mu)^2/2\sigma^2}
\end{align*}
\end{definition}

\section*{Linear Regression with Clamping}


\end{document}
